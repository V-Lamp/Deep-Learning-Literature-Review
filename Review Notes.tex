\documentclass[]{article}
\begin{document}
	\section{Deep Learning: Methods and Applications \cite{Deng2013}}
		It is a book, 197 pages.
		\subsection{Abstract}
			Deep Learning: Methods and Applications provides an overview of general deep learning methodology and its applications to a variety of signal and information processing tasks. The application areas are chosen with the following three criteria in mind: 
		\begin{itemize}
			\item(1) expertise or knowledge of the authors; 
			\item(2) the application areas that have already been transformed by the successful use of deep learning technology, such as speech recognition and computer vision; and 
			\item(3) the application areas that have the potential to be impacted significantly by deep learning and that have been benefitting from recent research efforts, including natural language and text processing, information retrieval, and multimodal information processing empowered by multi- task deep learning.
		\end{itemize}
		
	
	
	\bibliographystyle{}	
	\bibliography{library}	
	
\end{document}

